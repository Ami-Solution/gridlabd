\documentclass[12pt,letterpaper,twoside]{article}
\usepackage[utf8]{inputenc}
\usepackage[left=2cm,right=2cm,top=2cm,bottom=2cm]{geometry}
\usepackage{amsmath}
\usepackage{amsthm}
\usepackage{amssymb}
\usepackage{graphicx}
\usepackage{tikz}
\usepackage[all,cmtip]{xy}

\newcommand{\eq}[1]{\begin{align*}#1\end{align*}}
\newcommand{\eqn}[2]{\begin{align}\label{eq:#1}#2\end{align}}
\newcommand{\subeq}[1]{\begin{subequations}\eq{#1}\end{subequations}}
\newcommand{\subeqn}[2]{\begin{subequations}\eqn{#1}{#2}\end{subequations}}
\newcommand{\Matrix}[1]{\begin{bmatrix}#1\end{bmatrix}}	
\newcommand{\Array}[2]{\begin{array}{#1}#2\end{array}}
\newcommand{\Packed}[4]{\left[\begin{array}{c|c}{#1}&{#2}\\\hline{#3}&{#4}\end{}\right]}
\newcommand{\Diagram}[4]{\begin{figure}[#1]\centerline{\xymatrix{#4}}\caption{#3}\label{fig:#2}\end{figure}}
\renewcommand{\deg}[1]{$^\circ$#1}

\author{David P. Chassin}

\title{Powernet Building Model}

\begin{document}

\maketitle

\section{Introduction}

\Diagram{!b}{system}{Transactive control system diagram} 
{
		w \ar[rr]
	&
	&	*++[F]{Plant} 
		\ar[rr]
		\ar@<-0.2cm>`r[dr]`[d]^{u}[d]
	&
	&	z
	\\
	&	
	&	*++[F]{Controller} 
		\ar@<-0.2cm>`l[ul]`[u]^{y}[u]
		\ar@<0.2cm>`l[dl]`[d]_{v}[d]
	&
	\\	p \ar[rr]
	&
	&	*++[F]{Agent}
		\ar[rr]
		\ar@<0.2cm>`r[ur]`[u]_{r}[u]
	&
	&	b
}

The powernet module plant model is an arbitrary order linear model, as shown in Figure~\ref{fig:system}, defined by the linear system
\subeqn {fullmodel}
{
	\dot x &= A x + B_1 p + B_2 w
\\
	z &= C_1 x + D_{11} p + D_{12} w
\\
	b &= C_2 x + D_{21} p + D_{22} w	
}
or in standard form
\eqn {stdmodel}
{
	\Matrix { \dot x \\ z \\ b } =
	\left[ \begin{array}{c|c}
		\Array{c}{A}
	&	\Array{cc}{B_1 & B_2}
	\\ \hline
		\Array{c}{C_1 \\ C_2}
	&	\Array{cc}{D_{11} & D_{12} \\ D_{21} & D_{22}}
	\end{array} \right]
	\Matrix { x \\ w \\ p }
}
where 
\begin{itemize}

\item $x$ is the internal state vector,

\item $A$ is the plant response matrix, 

\item $B_1$ is the market input matrix, 

\item $p$ is the market input, 

\item $B_2$ is the disturbance input matrix, 

\item $w$ is the disturbance input, 

\item $z$ is the performance metric output,

\item $C_1$ is the performance output matrix,

\item $C_2$ is the bid output matrix,

\item $D_{11}$ is the feed-forward price performance matrix

\item $D_{12}$ is the feed-forward disturbance performance matrix

\item $D_{21}$ is the feed-forward price bid matrix

\item $D_{22}$ is the feed-forward disturbance bid matrix


\end{itemize}

\subsection{Residential buildings}

\Diagram{!b}{residential}{Residential thermal network diagram} 
{
	&	*+o[F]{T_A}
			\ar@{-}[dd]_{U_M}
			\ar@{=}[r]
	&	*++[F]{C_A}
\\		*+o[F]{T_O} 
			\ar@{-}[ru]^{U_A}
			\ar@{-}[rd]_{U_W}
	&
	&	*o[F]{} 
			\ar[ld]_{K_M} 
			\ar[lu]^{K_A}
	&	Q
			\ar@{-}[l]
			%\ar`l/0pt[l]`_ul[llu][llu]
			%\ar`l/0pt[l]`_dl[lld][lld]
\\
	&	*+o[F]{T_M}
			\ar@{=}[r]
	&	*++[F]{C_M}
}

We use the second-order thermal response model for a fully connect thermal network shown in Figure~\ref{fig:residential}.
\subeqn {residential_thermal_model} 
{
	U_A(T_O-T_A) + U_M(T_M-T_A)-C_A\dot T_A + n Q_N + K_1 Q_I + K_2 Q_S &= 0
\\
	U_M(T_A-T_M) + U_W(T_O-T_M)-C_M\dot T_M + K_3 Q_I + K_4 Q_S &= 0
}
where
\begin{itemize}

\item $U_A$ is the conductivity of the exterior envelope to the indoor air (in W/\deg K),

\item $T_O$ is the outdoor air temperature (in \deg C),

\item $T_A$ is the indoor air temperature (in \deg C),

\item $U_M$ is the conductivity of the building mass to the indoor air (in W/\deg K),

\item $T_M$ is the building mass temperature (in \deg C),

\item $C_A$ is the thermal capacity of the indoor air (in J/\deg K),

\item $n$ is the number of occupants (pu adults),

\item $Q_N$ is the heat gain from a single occupant in (in W/adult), 

\item $K_1$ is the fraction of internal heat gain that goes to the indoor air,

\item $Q_I$ is the internal heat gain (in W)

\item $K_2$ is the fraction of system heat that goes to the air,

\item $Q_S$ is the system power (in W),

\item $U_W$ is the conductivity of the building mass to the outdoor air (in W/\deg K)

\item $C_M$ is the thermal capacity of the building mass (in J/deg K), 

\item $K_3$ is the fraction of internal heat gain that goes to the building mass, and

\item $K_4$ is the fraction of system heat that goes to the building mass.

\end{itemize}
The internal heat gains are computed as 
\eq 
{
	Q_I &= P_A + \eta_I P_I
}
and the system heat gains are
\eq 
{
	Q_S &= m_e \eta_E P_S  + m_g \eta_H P_H
}
where
\begin{itemize}

\item $P_A$ is the electric power demand for all indoor appliances in (in W),

\item $\eta_G$ is the thermal heat for one unit of gas (in W/kg),

\item $P_I$ is the gas usage for all indoor appliances (in kg),

\item $m_e$ is the electric system mode (pu $Q_E$),

\item $m_g$ is the gas system mode (pu $Q_H$),

\item $\eta_E$ is the electric system efficiency (pu $P_S$),

\item $P_S$ is the electric system power demand (in W),

\item $\eta_I$ is the gas impact factor (pu $P_I$), and

\item $\eta_H$ is the gas heating efficiency (pu $P_H$), and

\item $P_H$ is the gas system consumption (in kg).

\end{itemize}
We rewrite (\ref{eq:residential_thermal_model}) as
\subeqn {residentialmodel} 
{
	\dot T_A = & -\frac{U_A+U_M}{C_A} T_A + \frac{U_M+U_W}{C_A} T_M + \frac{U_A}{C_A} T_O + nQ_N
		\\ & + \frac{K_A}{C_A}(P_A+n_G\eta_IP_I+m_E\eta_EP_S+m_G\eta_HP_H)
\\
	\dot T_M = & -\frac{U_M+U_W}{C_M} T_M + \frac{U_W}{C_M} T_O 
		\\ & + \frac{K_M}{C_M}(P_A+n_G\eta_IP_I+m_E\eta_EP_S+m_G\eta_HP_H)
}
with the total consumptions
\subeqn {residentialuse} 
{
	P_E &= P_A + P_S
\\
	P_G &= P_I + P_H.
}
The output bid price of electricity is 
\eq {
	b = k (T_A-T_D)
}
where 
\begin{itemize}
\item $k$ is the comfort setting, and 
\item $T_D$ is the desired indoor air temperature.
\end{itemize}
The output indoor temperature setpoint is
\eq {
	T_S = \frac{1}{k} p + T_D
}
where $p$ is the input price of electricity. The total costs of electric and gas consumption are
\subeqn {residentialcost}
{
	C_E &= p P_E
\\
	C_G &= g P_G,
}
respectively, where $g$ is the price of gas.

We now construct the system model for Eq.~(\ref{eq:stdmodel}) as follows. The state, input and output vectors are
\eq {
	x &= \Matrix { T_A \\ T_M }
&
	w &= \Matrix {T_O \\ n \\ m_e \\ m_g \\ P_A \\ P_I \\ T_D }
&
	z &= \Matrix { C_T \\ C_E \\ C_G },
}
where $C_T$ is the utility of comfort.
The system matrices for Eq.~(\ref{eq:stdmodel}) are
\eq {
		A &= \Matrix {
			-\frac{U_A+U_M}{C_A} & \frac{U_M+U_W}{C_A} 
		\\
			\frac{U_M}{C_M} & -\frac{U_M+U_W}{C_M} 
		}
	&
		B_1 &= \Matrix { 
			\frac{U_A}{C_A} & Q_N & \frac{K_A \eta_E P_S}{C_A} & \frac{K_A \eta_H P_H}{C_A} & \frac{K_A}{C_A} & \frac{K_A \eta_G}{C_A} & 0
		\\
			\frac{U_W}{C_M} & 0 & \frac{K_M \eta_E P_S}{C_M} & \frac{K_M \eta_H P_H}{C_M} & \frac{K_M}{C_M} & \frac{K_M \eta_G}{C_M} & 0
		}
	&
		B_2 & = \Matrix {
			0
		\\
			0
		}
	\\
		C_1 & = \Matrix {
			k & 0
	\\
			0 & 0
	\\
		 	0 & 0
		}
	&
		D_{11} & = \Matrix {
			0 & 0 & 0 & 0 & 0 & 0 & -k
	\\
			0 & 0 & pP_S & 0 & p & 0 & 0
	\\
		 	0 & 0 & 0 & gP_H & 0 & g & 0
		}
	&
		D_{12} & = \Matrix {
			-1
	\\
			0
	\\
			0
		}
	\\
		C_2 & = \Matrix {
			k & 0
		}
	&
		D_{21} & = \Matrix {
			0 & 0 & 0 & 0 & 0 & 0 & -k
		}
	&
		D_{22} & = \Matrix {
			0
		}
}

\end{document}